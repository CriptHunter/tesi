\documentclass[12pt,italian]{report}
\usepackage{tesi}

%
%			INFORMAZIONI SULLA TESI
%			DA COMPILARE!
%

% CORSO DI LAUREA:
\def\myCDL{Corso di Laurea magistrale in\\Informatica}
% TITOLO TESI:
\def\myTitle{Il titolo\\della tesi}

% AUTORE:
\def\myName{Lorenzo D'Alessandro}
\def\myMat{Matr. Nr. 939416}

% RELATORE E CORRELATORE:
\def\myRefereeA{Relatore 1}
\def\myRefereeB{Correlatore 1}

% ANNO ACCADEMICO
\def\myYY{2020-2021}

% Il seguente comando introduce un elenco delle figure dopo l'indice (facoltativo)
%\figurespagetrue

% Il seguente comando introduce un elenco delle tabelle dopo l'indice (facoltativo)
%\tablespagetrue

%
%			PREAMBOLO
%			Inserire qui eventuali package da includere o definizioni di comandi personalizzati
%

% Package di formato
\usepackage[a4paper]{geometry}		% Formato del foglio
\usepackage[italian]{babel}			% Supporto per l'italiano
\usepackage[utf8]{inputenc}			% Supporto per UTF-8
%\usepackage[a-1b]{pdfx}			% File conforme allo standard PDF-A (obbligatorio per la consegna)

% Package per la grafica
\usepackage{graphicx}				% Funzioni avanzate per le immagini
\usepackage{hologo}					% Bibtex logo with \hologo{BibTeX}
%\usepackage{epsfig}				% Permette immagini in EPS
%\usepackage{xcolor}				% Gestione avanzata dei colori

% Package tipografici
\usepackage{amssymb,amsmath,amsthm} % Simboli matematici
\usepackage{listings}				% Scrittura di codice

% Package ipertesto
\usepackage{url}					% Visualizza e rendere interattii gli URL
\usepackage{hyperref}				% Rende interattivi i collegamenti interni


\begin{document}

% Creazione automatica del frontespizio
\frontespizio
\beforepreface

% 
%			PAGINA DI DEDICA E/O CITAZIONE
%			facoltativa, questa è l'unica cosa che dovete formattare a mano, un po' come vi pare
%

{\raggedleft \large \sl Dedica \\}
         
% 
%			PREFAZIONE (facoltativa)
%

%\prefacesection{Prefazione}
%Le prefazioni non sono molto comuni, tuttavia a volte capita che qualcuno voglia dire qualcosa che esuli dal lavoro in s\'e (come un meta-commento sull'elaborato), o voglia fornire informazioni riguardanti l'eventuale progetto entro cui la tesi si colloca (in questo caso \`e probabile che sia il relatore a scrivere questa parte).

%
%			RINGRAZIAMENTI (facoltativi)
%

\prefacesection{Ringraziamenti}
Questa sezione, facoltativa, contiene i ringraziamenti.

%
%			Creazione automatica dell'indice
%

\afterpreface

% 
%			CAPITOLO 1: Introduzione o Abstract
% 

\chapter{Introduzione}
\label{cap:introduzione}

Introduzione...

\section{I contenuti}
\label{sec:contenuti}

Spiegazione problema...


\section{Organizzazione della tesi}
\label{sec:organizzazione}

Organizzazione tesi...

% 
%			CAPITOLO 2: Stato dell'arte
% 

\chapter{Stato dell'arte}
\label{chap:stato_arte}



% 
%			CAPITOLO 3: Lavoro svolto
% 

\chapter{Classificatore}
\label{chap:classificatore}

I recommender system della famiglia collaborative filtering utilizzano le informazioni su users e items per raccomandare gli items a users simili...
Il numero di users e items è fisso, questo è vero sia per matrix factorization che lavora su una matrice con un numero di righe pari al numero di users e un numero di colonne pari al numero di items, sia per i modelli neurali che utilizzano dei vettori di embedding con dimensione decisa a tempo di compilazione. Aggiungere un nuovo user/item significa ricompilare il modello ed eseguire nuovamente la fase di training. Questo non è un problema in un ambiente desktop in cui dei server aggiornano giornalmente le preferenze di ogni utente. Diverso è il caso in cui si vuole eseguire il training del recommender system direttamente sul dispositivo mobile, ed è quindi impensabile eseguire il training ogni volta che si incontra un nuovo user/item

Il modello proposto in questa tesi è una rete feed-forward fully-connected, un modello di rete neurale molto comune. Questa rete è composta da $n$ layers che contengono $x$ neuroni. Ogni neurone è connesso a tutti i neuroni del livello successivo e non esistono cicli nel grafo \cite{Goodfellow-et-al-2016}


% 
%			CAPITOLO 4: Datasets
% 

\chapter{Datasets}
\label{chap:datasets}


% 
%			CAPITOLO 5: Risultati
% 

\chapter{Risultati}
\label{chap:risultati}


% 
%			CAPITOLO 6: Conclusioni e sviluppi futuri
% 

\chapter{Conclusioni}
\label{cap6}

\section{Conclusioni}

Conclusioni...

\section{Sviluppi futuri}

Sviluppi futuri...



%
%			BIBLIOGRAFIA
%

\bibliographystyle{unsrt}
\bibliography{bibliografia}
\nocite{*}
\addcontentsline{toc}{chapter}{Bibliografia}


\end{document}


 
